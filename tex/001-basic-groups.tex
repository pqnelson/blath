\N{Introduction} We will do some group theory. Here ``group'' refers to
a ``group of symmetry transformations'', and we should think of elements
of the group as functions mapping an object to itself in some
particularly \emph{symmetric} way.

\begin{definition}
  A \define{Group} consists of a set $G$ equipped with
  \begin{enumerate}
  \item a law of composition $\circ\colon G\times G\to G$,
  \item an identity element $e\in G$, and
  \item an inverse operator $(-)^{-1}\colon G\to G$
  \end{enumerate}
  such that
  \begin{enumerate}
  \item Associativity: For any $g_{1}$, $g_{2}$, $g_{3}\in G$, $(g_{1}\circ g_{2})\circ g_{3}=g_{1}\circ(g_{2}\circ g_{3})$
  \item Unit law: For any $g\in G$, $g\circ e=e\circ g=g$
  \item Inverse law: For any $g\in G$, $g^{-1}\circ g=g\circ g^{-1}=e$.
  \end{enumerate}
\end{definition}

\N{Effective Thinking Principle: Create Examples.} Whenever encountering
a new definition, it's useful to construct examples. Plus, it's fun. Now
let us consider a bunch of examples!

\begin{example}[Trivial]
One strategy is to find the most boring example possible. We can't use
$G=\emptyset$ since a group must contain at least one element: the
identity element $e\in G$. Thus the next most boring candidate is the
group containing \emph{only} the identity element $G=\{e\}$. This is the
\define{Trivial Group}.
\end{example}

\begin{example}[Dihedral]
Consider the regular $n$-gon in the plane $X\subset\RR^{2}$ with
vertices located at $(\cos(k2\pi/n), \sin(k2\pi/n))$ for
$k=0,1,\dots,n-1$. We also require $n\geq3$ to form a non-degenerate
polygon ($n=2$ is just a line segment, and $n=1$ is one dot).

We can rotate the polygon by multiples of $2\pi/n$ radians. There are
several ways to visualize this, I suppose we could consider rotations of
the plane by $2\pi/n$ radians:
\begin{equation}
  r\colon\RR^{2}\to\RR^{2}
\end{equation}
which acts like the linear transformation
\begin{equation}
  r
  \begin{pmatrix}
    x\\
    y
  \end{pmatrix}
  :=
  \begin{pmatrix}
    \cos(2\pi/n) & -\sin(2\pi/n)\\
    \sin(2\pi/n) & \cos(2\pi/n)
  \end{pmatrix}
  \begin{pmatrix}
    x\\
    y
  \end{pmatrix}.
\end{equation}
We see that the image of our $n$-gon under this transformation $r(X)=X$
remains invariant.

The other transformation worth exploring is reflecting about the
$x$-axis, $s\colon\RR^{2}\to\RR^{2}$ which may be defined by
\begin{equation}
  s
  \begin{pmatrix}
    x\\
    y
  \end{pmatrix}
  :=
  \begin{pmatrix}
    1 & 0\\
    0 & -1
  \end{pmatrix}
  \begin{pmatrix}
    x\\
    y
  \end{pmatrix}.
\end{equation}
This transformation also leaves our polygon invariant $s(X)=X$.

We can compose these two types of transformations. Observe that
$s\circ s=\id$ and the $n$-fold composition $r^{n}=r\circ\dots\circ r=\id$
both yield the identity transformation $\id(x)=x$ for all $x\in\RR^{2}$.
Then we have $2n$ symmetry transformations: $\id$, $r$, ..., $r^{n-1}$;
and $s$, $s\circ r$, ..., $s\circ r^{n-1}$. What about, say, $r\circ s$?
We find $s\circ r^{k}\circ s=r^{-k}$, so $r^{k}\circ s = s\circ r^{-k}$.
Thus it's contained in our list of symmetry transformations.

The symmetry group thus constructed is called the \define{Dihedral Group}.
Geometers denote it by $D_{n}$, algebraists denote it by $D_{2n}$, and
we denote it by $\dihedral{n}$.
\end{example}

\begin{example}[Rotations of regular polygon]
We can restrict our attention, working with the previous example
further, to only \emph{rotations} of the regular $n$-gon by multiples of
$2\pi/n$ radians. We can describe this group as ``generated by a single element'',
i.e., symmetries are of the form $r^{k}$ for $k\in\ZZ$. This is an
example of a \define{Cyclic Group}. In particular, it is commutative:
any symmetries $r_{1}$ and $r_{2}$ satisfy $r_{1}\circ r_{2}=r_{2}\circ r_{1}$.
These are special situations, let us carve out space to define these
concepts explicitly.
\end{example}

\begin{definition}
  We call a group $G$ \define{Abelian} if it is commutative, i.e.,
  for any transformations $f$,
  $g\in G$ we have $f\circ g = g\circ f$. In this case, we write $f\circ g$
  as $f+g$, using the plus sign to stress commutativity.
\end{definition}

\begin{definition}
  We call a group $G$ \define{Cyclic} if there is at least one element
  $g\in G$ such that $\{g^{n}\mid n\in\ZZ\}=G$ the entire group consists
  of iterates of $g$ and $g^{-1}$.
\end{definition}

\begin{example}[Number Systems]
Another few examples the reader may know are the familiar number systems
under addition: the integers $\ZZ$, the rational numbers $\QQ$, the real
numbers $\RR$, and the complex numbers $\CC$. They are commutative groups.
\end{example}

\begin{exercise}
 Is $\ZZ$ a cyclic group? Is $\CC$ a cyclic group?
\end{exercise}
\begin{exercise}
  Is the non-negative integers $\NN_{0}$ a group under addition? Under multiplication?
\end{exercise}
\begin{exercise}
  Are the positive real numbers $\RR_{\text{pos}}$ a group under multiplication?
\end{exercise}

\begin{example}[Infinite dihedral]
We can take the infinite limit of the dihedral group to get the infinite
dihedral group $\dihedral{\infty}$. We formally describe it as
consisting of ``rotations'' $r$ and ``reflections'' $s$ such that
\begin{enumerate}
\item $r^{m}\circ r^{n} = r^{m+n}$ for any $m$, $n\in\ZZ$;
\item $s\circ r^{m}\circ s = r^{-m}$ for any $m\in\ZZ$;
\item $s\circ s = e$;
\item $r^{n}\circ r^{-n} = r^{-n}\circ r^{n} = e$ for any $n\in\ZZ$, in
  particular $r^{0}=e$.
\end{enumerate}
In this sense, the ``infinite limit'' turns rotations into something
like the integers.
\end{example}

\begin{example}[Circular dihedral]
A more intuitive ``infinite limit'' of the dihedral group is the
symmetries of the unit circle $S^{1}$ in the plane $\RR^{2}$. These are
anti-clockwise rotations and reflection about the $x$-axis, but
rotations are parametrized by a real parameter (the ``angle''):
\begin{equation}
  r_{\theta} \mbox{``=''} \begin{pmatrix}\cos(\theta) & -\sin(\theta)\\
    \sin(\theta) & \cos(\theta)
  \end{pmatrix}.
\end{equation}
Here we write an ``equals'' sign in quotes because this is the
intuition. A group is \emph{abstract}, whereas the matrix is a concrete
realization of the symmetry.

The reader should verify the axioms for a group are satisfied, with the
hint that $r_{\theta}\circ r_{\phi} = r_{\theta+\phi}$ and the usual
relation between reflection and rotation holds.

This group is called the \define{Orthogonal Group} in 2-dimensions.
\end{example}

\begin{exercise}
Pick your favorite polyhedron in 3-dimensions. Determine its symmetry group.
\end{exercise}

\begin{exercise}
Complex conjugation acts on $\CC$ by sending $x+i\cdot y$ to $x-i\cdot y$.
Does this give us a symmetry group?
\end{exercise}

\begin{exercise}
If we consider polynomials with coefficients in, say, rational numbers
(denoted $\QQ[x]$ for polynomials with the unknown $x$), then how can we
form a symmetry group of $\QQ[x]$?
\end{exercise}

\begin{exercise}[General Linear Group]
  Take $n\in\NN$ to be a fixed positive integer, preferably $n\geq2$.
  Consider the collection of invertible $n$-by-$n$ matrices with entries
  which are rational numbers
  $$\GL(n, \QQ) = \{ M\in\Mat(n\times n, \QQ) \mid \det(M)\neq0\}.$$
  Prove this is a group under matrix multiplication.
\end{exercise}

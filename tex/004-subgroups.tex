\begin{definition}
  Let $G$ be a group; denote its group structure as the tuple
  $(G,e_{G},\star_{G},\iota_{G})$ where $\iota_{G}$ is the inverse
  operator, $e_{G}$ is the identity element, $\star_{G}$ is the binary
  composition operator.

  We define a \define{Subgroup} of $G$ to consist of a subset $H\subseteq G$
  equipped with $G$'s group structure, i.e.,
  \begin{enumerate}
  \item the binary operation restricted to $H$, $\star|_{H}\colon H\times H\to G$
  \item the inverse operator $\iota|_{H}\colon H\to H$
  \item the identity element from $G$, $e_{G}\in H$
  \end{enumerate}
  such that
  \begin{enumerate}
  \item the group composition is closed on $H$, i.e.,
    $\{h_{1}\star|_{H}h_{2}|h_{1},h_{2}\in H\}\subseteq H$
  \item group inversion is closed in $H$, i.e., $\iota|_{H}(H)\subseteq H$
  \item the usual group properties holds for this group structure
    induced on $H$.
  \end{enumerate}
  If further $H\neq G$, we call $H$ a \define{Proper Subgroup} of $G$.
\end{definition}

\begin{remark}
  We denote $H<G$ if $H$ is a proper subgroup of $G$, and $H\leq G$ if
  $H$ is a generic subgroup of $G$.
\end{remark}

\begin{example}[Trivial subgroups]
  For any group $G$, there are always two subgroups available: $G$
  itself, and the trivial subgroup $\mathbf{1}$ consisting of only the
  identity element. Since these subgroups come ``for free'', we call
  them \define{Trivial Subgroups}.
\end{example}

\begin{example}[Subgroups of dihedral group]
  Let $n\in\NN$ be greater than 2. Recall the dihedral group
  $\dihedral{n}$ consists of rotations by $2\pi/n$ radians and
  reflections.
  We have two subgroups (at least): one generated by rotations alone,
  the other generated by reflections alone.

  The subgroup of rotations is a finite group consisting of $n$
  elements. It's a cyclic group, isomorphic to $\ZZ/n\ZZ$.

  The subgroup of reflections is less exciting. There are two elements
  in it: reflection about the $x$-axis, and the identity
  transformation. This is a cyclic subgroup of order 2, isomorphic to
  $\ZZ/2\ZZ$. 
\end{example}

\begin{example}[NON-Example: subset alone insufficient]
  Consider modular arithmetic $\ZZ/p\ZZ$ for some prime $p$. This is a
  group consisting of a set $\{0,1,\dots,p-1\}$ equipped with addition
  modulo $p$, inversion maps $m$ to $-m\equiv p-m\bmod{p}$.

  Although the underlying set is a subset of the additive group $\ZZ$,
  the addition operation is a different function. Consequently, the
  group structure on $\ZZ/p\ZZ$ is \emph{not equal} to the group
  structure on $\ZZ$. The moral: being a subset alone is insufficient.
\end{example}

\begin{example}[Symmetric group]
Let $n\in\NN$. The symmetric group $S_{n}$ consists of permutations of
the set $\Omega_{n}=\{1,2,\dots,n\}$. We can consider $k\leq n$, and
then we have $S_{k}\leq S_{n}$ by restricting to permutations of the
subset $\Omega_{k}\subseteq\Omega_{n}$.
\end{example}

\begin{example}
Recall the general linear group $\GL(n,\RR)$ is the group of $n\times n$
invertible matrices over the real numbers, using matrix multiplication
as its binary operator, and matrix inversion for its group inverse. Also
recall the determinant
\begin{equation}
  \det\colon\GL(n,\RR)\to\RR^{\times}
\end{equation}
is a group morphism. Its kernel is a subgroup, called the
\define{Special Linear Group}, denoted
\begin{equation}
\SL(n,\RR)=\{M\in\GL(n,\RR)|\det(M)=1\}.
\end{equation}
The reader can check it is closed under matrix multiplication and matrix
inversion. 
\end{example}

\begin{example}\label{ex:004-subgroups:commutator-subgroup}
Let $G$ be any group. If $g_{1},g_{2}\in G$, then their commutator is
the element $[g_{1},g_{2}]=g_{1}g_{2}g_{1}^{-1}g_{2}^{-1}\in G$.
We can construct the \define{Commutator Subgroup} of $G$, denoted
$[G,G]$ or (more commonly but more confusingly) by $G'$ in the literature,
by considering the subgroup generated by commutators of elements of $G$.
Then an arbitrary element of the commutator subgroup looks like
\begin{equation}
{}  [g_{1},g_{2}][g_{3},g_{4}](\dots)[g_{2n-1},g_{2n}]\in[G,G]
\end{equation}
and we use multiplication from $G$.
\end{example}

\section{Exercises}

\begin{exercise}
  Let $H_{1}$, $H_{2}\leq G$. Is $H_{1}\cap H_{2}$ a subgroup of $G$?
\end{exercise}

\begin{exercise}
  Let $H_{1}$, $H_{2}\leq G$. Is $H_{1}\cup H_{2}$ a subgroup of $G$?
\end{exercise}

\begin{exercise}
Let $\varphi\colon G\to K$ be a group morphism. Is the image of
$\varphi$ a subgroup of $K$?

If $H<K$, is its preimage under $\varphi$ a subgroup of $G$?
\end{exercise}

\begin{exercise}
Prove or find a counter-example: if $G$ is a cyclic group, then every
subgroup is cyclic.
\end{exercise}

\section{Normal Subgroups}

\M\label{chunk:004-subgroups:motivation-for-normal-subgroup}
The special linear group is a rather special situation (no pun intended).
We can prove that, for any $M\in\SL(n,\RR)$ and for any
$T\in\GL(n,\RR)$, we have $XMX^{-1}\in\SL(n,\RR)$. Really? Look, take
its determinant:
\begin{equation}
  \begin{split}
    \det(XMX^{-1})&=\det(X)\det(M)\det(X)^{-1}\\
    &=\det(M)=1.
  \end{split}
\end{equation}
Its a property shared by the kernel of any group morphism
$\varphi\colon G\to H$, we'd have $k\in\ker(\varphi)$ and $g\in G$
satisfy $\varphi(gkg^{-1})=\varphi(g)\varphi(k)\varphi(g)^{-1}=\varphi(k)=e_{H}$.
This gives us a particular kind of subgroups.

\begin{definition}\label{defn:004-subgroups:normal-subgroup}
Let $G$ be a group. A \define{Normal Subgroup} of $G$ consists of a
subgroup $H\leq G$ such that
\begin{enumerate}
\item Closed under conjugation by group elements: for any $h\in H$ and for any $g\in G$, we have $ghg^{-1}\in H$.
\end{enumerate}
\end{definition}

\begin{remark}
Care must be taken for the normality property: all we ask is for
$ghg^{-1}\in H$, \textbf{not} that $ghg^{-1}=h$. 
\end{remark}

\begin{remark}[Notation]
  We denote a normal subgroup $N$ of $G$ by $N \normalSubgroup G$, and
  $N \properNormalSubgroup G$ if $N$ is a proper normal subgroup of
  $G$. There are a couple ways of remembering which way the triangle
  points: one is that we just turn the subgroup notation $N\leq G$ into
  a triangle, the other is that it ``snitches'' on (i.e., points to) the
  normal subgroup.
\end{remark}

\begin{example}
The trivial subgroups are trivially normal.
\end{example}

\begin{example}
For the Dihedral group $\dihedral{n}$, we have a couple subgroups (one
generated by rotations, the other generated by reflections). Is one of
them normal?

The rotation subgroup is generated by $r$ and satisfies $r^{n}=1$. The
reflection subgroup is generated by $s$ and satisfies
$s^{2}=1$. Together, they have $s\circ r^{k}\circ s=r^{-k}$. This
implies the rotation subgroup is normal.

Is the reflection subgroup normal?
\end{example}

\begin{theorem}\label{thm:004-subgroups:abelian-implies-subgroups-are-normal}
  Let $G$ be a group.
  If $G$ is Abelian, then every subgroup $H\leq G$ is normal.
\end{theorem}

\begin{proof}
Assume $G$ is Abelian. Let $H\leq G$ be an arbitrary subgroup of $G$,
let $g\in G$ and $h\in H$ be arbitrary group elements. Then conjugation
looks like $g+h-g=h$. Hence $H$ is a normal subgroup of $G$ by
Definition~\ref{defn:004-subgroups:normal-subgroup}.
\end{proof}

\begin{example}[{Converse of Theorem~\ref{thm:004-subgroups:abelian-implies-subgroups-are-normal} is false}]
Consider the quaternion group $Q_{8}$ generated by
$i^{4}=j^{4}=k^{4}=(ijk)^{2}=1$ and % $ij=-ji=k$, $jk=-kj=i$, $j=ki=-ik$.
so $ij=k$, $jk=i$, $j=ki$. We really have only 8 elements in $Q_{8}$:
$1$, $-1$, $i$, $-i$, $j$, $-j$, $k$, $-k$. But we only need two of the
``purely imaginary'' elements ($i$, $j$, $k$) to generate the entire group.

The proper subgroups would be generated from one complex generator, or
from $-1$. We claim they are all normal. The case of $\{\pm1\}<Q_{8}$ is
obviously normal, since $x(-1)x^{-1}=-1$ for any $x\in Q_{8}$.

The reasoning for proper subgroups generated by one imaginary element
resembles one another, so let's consider the subgroup generated by $i$.
Then we see
\begin{equation}
jij^{-1}=-jk=-i
\end{equation}
and 
\begin{equation}
  j^{3}ij^{-3}=(-j)i(j)=i.
\end{equation}
Similarly we'd find, from $i^{3}=-i$, that
\begin{equation}
  ji^{3}j^{-1}=i
\end{equation}
and
\begin{equation}
  j^{3}i^{3}j^{-3}=-i.
\end{equation}
Thus the subgroup generated by $i$ is closed under conjugation from all
elements from $Q_{8}$.

Really? Well, any generic element in $Q_{8}$ looks like
$i^{m}j^{n}$. Its inverse would be $(i^{m}j^{n})^{-1}=j^{-n}i^{-m}$. So
conjugation by these elements would amount to multiplying the element
$i^{\ell}$ from the subgroup by some $i^{q}$. But that's okay: elements
of the form $i^{q}$ belong to the subgroup anyways.
\end{example}

\begin{theorem}
  If $\varphi\colon G\to H$ is a group morphism,
  then $\ker(\varphi)$ is a normal subgroup of $G$.
\end{theorem}

%\bigskip
\noindent The proof was sketched earlier in \S\ref{chunk:004-subgroups:motivation-for-normal-subgroup}.

\begin{theorem}
  Let $G$ be a group.
  Every normal subgroup $N\normalSubgroup G$ is the kernel of some group
  morphism. 
\end{theorem}

\begin{remark}
We do not yet have the technology to prove this, yet, but it is true.
Basically, we construct a morphism by taking $g\in G$ and mapping it to
cosets $gN=\{gn\in G|n\in N\}$. The difficulty we have is that the image
of this mapping is the collection of left cosets of $N$, which may or
may not be a group. Further, it was rather arbitrary mapping it to
\emph{left} cosets of $N$: why not map $g$ to $Ng=\{ng\in G|n\in N\}$?
Wouldn't these be distinct (i.e., not equal) mappings?
\end{remark}

\section{Exercises}

\begin{exercise}
Find the normal subgroups of the symmetric group $S_{n}$.
\end{exercise}

\begin{exercise}
Let $G$ be a group, $H\normalSubgroup G$ be a normal subgroup, and
$N\normalSubgroup H$ be a normal subgroup of $H$. Is $N$ a normal
subgroup of $G$?
\end{exercise}

\begin{exercise}
Let $G$ be a group, recall the commutator subgroup from Example~\ref{ex:004-subgroups:commutator-subgroup}
that $[G,G]$ is a subgroup of $G$. Prove or find a counter-example: the
commutator subgroup is a normal subgroup of $G$.
\end{exercise}

\begin{exercise}
Let $\varphi\colon G\to H$ be a group morphism, $N\normalSubgroup G$ be
a normal subgroup. Is $\varphi(N)$ a normal subgroup of $G$? Consider
the cases when $\varphi$ is injective, and when $\varphi$ is surjective.
\end{exercise}

\begin{exercise}
Let $\varphi\colon G\to H$ be a group morphism, and
$N_{H}\normalSubgroup G$ be a normal subgroup. Is the preimage
$\varphi^{-1}(N_{H}) = \{g\in G|\varphi(g)\in N_{H}\}$ a normal subgroup
of $G$? Is it even a subgroup?
\end{exercise}


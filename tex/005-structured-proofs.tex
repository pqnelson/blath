\M
Suppose we have successfully formalized the classification theorem for
finite simple groups in your favorite theorem prover (Lean, HOL, Coq,
Mizar, whatever). In a century's time, do you think your theorem prover
will still be around and backwards compatible? If not, what good would
the formalization be?

\N{Readable Proofs}
One of the several roles for proofs in mathematics is
\emph{communication}. Hence we need something analogous to the notion of
``readable code'' in programming: readable proofs. Consider the
following proof, tell me what it does (no cheating!):

\begin{Verbatim}
prove
 (`!p q. p * p = 2 * q * q ==> q = 0`,
  MATCH_MP_TAC num_WF THEN REWRITE_TAC[RIGHT_IMP_FORALL_THM] THEN
  REPEAT STRIP_TAC THEN FIRST_ASSUM(MP_TAC o AP_TERM `EVEN`) THEN
  REWRITE_TAC[EVEN_MULT; ARITH] THEN REWRITE_TAC[EVEN_EXISTS] THEN
  DISCH_THEN(X_CHOOSE_THEN `m:num` SUBST_ALL_TAC) THEN
  FIRST_X_ASSUM(MP_TAC o SPECL [`q:num`; `m:num`]) THEN
  POP_ASSUM MP_TAC THEN CONV_TAC SOS_RULE);;
\end{Verbatim}

Or the following:

\begin{Verbatim}
Proof.
  rewrite EL2 with (p_prime := p_prime) (p_odd := p_odd) by (omega || auto).
  rewrite EL2 with (p_prime := q_prime) (p_odd := q_odd) by (omega || auto).
  rewrite m1_pow_compatible. rewrite <- m1_pow_morphism.
  f_equal. rewrite <- QR6. unfold a. unfold b. auto.
  rewrite <- Zmult_0_r with (n := (p - 1) / 2).
  apply Zmult_ge_compat_l. omega. omega.
Qed.
\end{Verbatim}

I honestly wouldn't guess the first example had to do with proving
$\sqrt{2}\notin\QQ$, nor would I have imagined the second had to do with
the law of quadratic recipricocity. (The first in HOL Light, the second
in Coq.) What do you suppose mathematicians of the 25th century would
think of these proofs? Would our efforts formalizing mathematics in a
theorem prover yield fruit akin to Euclid's \emph{Elements}, or would it
be more indecipherable than Russell and Whitehead's \emph{Principia}?

Hence, I would like to assess the following question:

\begin{puzzle}
What qualities make a proof ``readable''?
\end{puzzle}

\N{Sources}
We're not in the wilderness on this matter. Some very smart people have
thought very hard about this question. But we haven't progressed much in
the past few decades, compared to designing programming languages. We
all seem to draw waters from the same well.

Andrzej Trybulec created Mizar~\ref{matuszewski:mizar} specifically for
readability. Over the years, Mizar went from an abstract idea to a
concrete realization, iteratively improving its input language to make
it resemble a working mathematician's proofs. A cynical critic may
deride Mizar as merely natural deduction in Pascal syntax, but that
concedes the point: it's readable and understandable on its own, without
relying on the computer to tell us the ``proof state'' and remaining
goals in a proof.

The other major source of inspiration, I think, is Mark Wenzel's Isar
front-end for Isabelle~\ref{wenzel:phd-thesis} (and enlightening
discussion of a prototype~\ref{wenzel:tphols99}). Here the system
\emph{emerges} bottom-up, as discussed in \S3.2 of Wenzel's
thesis~\ref{wenzel:phd-thesis}. Other theorem provers attempted to
simply copy/paste Isar commands, but apparently they never caught on. I
suspect this is due to copying the superficialities, rather than drawing
upon the underlying principles, produced an awkward input language.

There are other sources worth perusing, like Zammit's ``Structured Proof
Language''~\ref{zammit:phd-thesis} among others. We also mention Isar
inspired many prototypes to adapt Mizar-style proofs to other theorem
provers (Corbineau's Czar~\ref{corbineau-czar}, Giero and Wiedijk's MMode~\ref{giero-wiedijk-mmode},
Harrison's Mizar Mode for HOL~\ref{harrison-mizar-mode})
Also worth mentioning is \textsc{ForTheL}~\ref{paskevich-forthel} which
emerged from the Soviet SAD theorem prover~\ref{lyaletski-sad}, possibly
a future input language for Lean.

\N{Terminology}
One bit of terminology, John Harrison~\ref{harrison:proof-style} calls
the structured proof style ``declarative style proofs'', which seems
fine. Some fanatics of the procedural camp dislike the term. Giero and
Wiedijk~\ref{giero-wiedijk-mmode} point out the
differences between procedural and declarative styles are:
\begin{enumerate}
\item a procedural proof generally run backward (from goal to the
  assumptions), whereas declarative proofs run mostly forwards (from the
  assumptions to the goal);
\item a procedural proof generally does not have statements containing
  the statements which occur in the proof states, whereas declarative
  proofs do;
\item a procedural proofs have few proof cuts, whereas declarative
  proofs have nearly one cut for each proof step.
\end{enumerate}
It turns out these qualities \emph{do} make a difference on the
readability of a proof script. But this is a bit like trying to learn an
author's writing style by examining the statistics of the author's
writing, like ``Average number of words per sentence'' or ``Average
number of syllables per word''.

\section{Bibliography}

\begin{enumerate}[label={[\arabic*]},left=0pt]
\item\label{corbineau-czar} Pierre Corbineau,
  ``A Declarative Language For The Coq Proof Assistant''.
  In Marino Miculan, Ivan Scagnetto, and Furio Honsell, editors, \emph{Types for Proofs and Programs, International Conference, TYPES 2007, Cividale des Friuli, Italy, May 2-5, 2007, Revised Selected Papers},
  volume 4941 of Lecture Notes in Computer Science, pages 69--84. Springer, 2007.
  \url{https://kluedo.ub.uni-kl.de/frontdoor/deliver/index/docId/2100/file/B-065.pdf}
\item\label{giero-wiedijk-mmode} Mariusz Giero and Freek Wiedijk,
  ``MMode: A Mizar Mode for the proof assistant Coq''.
  Technical report, ICIS, Radboud Universiteit Nijmegen, 2004.
  \url{https://www.cs.ru.nl/~freek/mmode/mmode.pdf}
\item\label{harrison-mizar-mode} John Harrison,
  ``A Mizar Mode for HOL''.
  In \emph{Proceedings of the 9th International Conference on Theorem Proving in Higher Order Logics, TPHOLs '96}, volume
  1125 of LNCS, pages 203--220. Springer, 1996.
  \url{https://www.cl.cam.ac.uk/~jrh13/papers/mizar.html}
\item\label{harrison:proof-style} John Harrison,
  ``Proof Style''.
  \url{https://www.cl.cam.ac.uk/~jrh13/papers/style.html}
\item Stanislaw Ja\'skowski
  ``On the Rules of Suppositions in Formal Logic''.
  \emph{Studia Logica} \textbf{1} (1934) pp.5--32.
  \url{https://www.logik.ch/daten/jaskowski.pdf}
\item\label{lyaletski-sad} Alexander Lyaletski,
  ``Evidence  Algorithm  Approach  to  Automated  Theorem  Proving 
  and SAD Systems''.
  \url{http://ceur-ws.org/Vol-2833/Paper_14.pdf}
\item\label{matuszewski:mizar} Roman Matuszewski, Piotr Rudnicki,
  ``Mizar: The First 30 Years''.
  \emph{Mechanized Mathematics and Its Applications} \textbf{4}, 1 (2005) pp.3--24.
  \url{http://mizar.org/people/romat/MatRud2005.pdf}
\item\label{paskevich-forthel} Andrei Paskevich,
  ``The syntax and semantics of the ForTheL language''.
  Manuscript dated December 2007,
  \url{http://nevidal.org/download/forthel.pdf}
\item\label{wenzel:tphols99} Markus Wenzel,
  ``Isar --- a Generic Interpretative Approach to Readable Formal Proof
  Documents''.
  In Y. Bertot, G. Dowek, A. Hirschowitz, C. Paulin, L. Thery, editors, \emph{Theorem Proving in Higher Order Logics, TPHOLs'99}, LNCS 1690, \copyright Springer, 1999.
  \url{https://www21.in.tum.de/~wenzelm/papers/Isar-TPHOLs99.pdf}
\item\label{wenzel:phd-thesis} Markus Wenzel,
  ``Isabelle/Isar --- a versatile environment for human-readable formal
  proof documents''.
  PhD thesis,
  Institut f\"ur Informatik, Technische Universit\"at M\"unchen, 2002. 
  \url{https://mediatum.ub.tum.de/doc/601724/601724.pdf}
\item Markus Wenzel and Free Wiedijk.
  ``A comparison of the mathematical proof languages Mizar and Isar''.
  \emph{Journal of Automated Reasoning} \textbf{29} (2002), pp.389--411. 
\item\label{zammit:phd-thesis} Vincent Zammit,  (1999)
  ``On the Readability of Machine Checkable Formal Proofs''.
  PhD Thesis, Kent University, 1999.
  \url{https://kar.kent.ac.uk/21861/}
\end{enumerate}
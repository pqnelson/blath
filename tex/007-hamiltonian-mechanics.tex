\title{Summary of Hamiltonian Mechanics}
\author{Alex Nelson\footnote{Email: \texttt{pqnelson@gmail.com}\qquad
  Updated: \today}}
\date{September 10, 2022}
\maketitle

\section{Introduction}

\N{Starting point: Lagrangian mechanics}
We recall Lagrangian mechanics starts with a Lagrangian
$L(q^{1},\dots,q^{n},v^{1},\dots,v^{n})$ which is a function of $2n$
variables ($n$ position variables $q^{j}$ and associated with them some
$n$ velocity variables $v^{j}$). The Euler--Lagrange equations of motion
are then given by:
\begin{equation}
\frac{\D}{\D t}\frac{\partial L}{\partial v^{j}} - \frac{\partial L}{\partial q^{j}} = 0
\end{equation}
where $j=1,\dots,n$. This is completely equivalent to the Newtonian
equations of motion. We can see this by writing
\begin{equation}
p_{j} := \frac{\partial L}{\partial v^{j}},
\end{equation}
which is the \define{Conjugate Momentum} (or \emph{generalized momentum},
and later \emph{canonical momentum}). Then
\begin{equation}
\frac{\D}{\D t}p_{j} - \frac{\partial L}{\partial q^{j}} = 0
\end{equation}
which is precisely Newton's second Law.

\M
Only now, after we get the equations of motion, do we replace all
instances of $q^{j}$ with components of a trajectory $\gamma(t)^{j}$ and
$v^{j}$ with its first time derivative $\dot\gamma(t)^{j}$. The
important point being we only do this after plugging in our Lagrangian
into the Euler--Lagrange equation of motion, and doing the partial
derivatives.

Then we end up with a second-order differential equation, which is hard,
and forces us to rue the day we ever attempted this endeavour. We
naturally ask: \emph{can we avoid second-order differential equations?}

\section{Hamiltonian Mechanics}

\M
I am torn between discussing Hamiltonian mechanics \emph{conceptually},
and just giving the steps involved in ``using Hamiltonian mechanics''.
In either case, we begin with Lagrangian mechanics as the starting
point. When doing Hamiltonian mechanics, we have already written down a
Lagrangian $L(\vec{q},\vec{v})$.

We can think of the Lagrangian as a function on the phase space --- a
hypothetical, mathematical space describing all possible ``physical
states'' of the system. But Lagrangian mechanics labels the coordinates
of this ``phase space'' using the positions $\vec{q}$ and velocities $\vec{v}$.
\emph{Besides this, the physical intuition remains the same as with Lagrangian mechanics.}
Among all possible trajectories, we are trying to find the ones which
are physically realizable.

We can therefore ``piggieback'' off all the hard work with Lagrangian
mechanics, the variational derivation of the equations of motion, then
make some last minute changes to pivot to the ``new coordinates'' of the
phase space. This is how most textbooks introduce Hamiltonian mechanics.
The flaw with this approach: we require velocities to be variables at
certain times, but at other times we require it to be a function of
position and momentum. It will appear random and unmotivated.

The other approach is slightly worse: introduce a new gadget called the
Poisson bracket, and use it to obtain the equations of motion.

\N{Step one: determine conjugate momentum}
We write down the conjugate momentum
\begin{equation}
p_{j}(\vec{q},\vec{v}) := \frac{\partial L(\vec{q},\vec{v})}{\partial v^{j}}.
\end{equation}
Right now, we think of it as just another collection of $n$ functions on
phase space.

\N{Step two: change coordinates}
We then change coordinates from $(\vec{q},\vec{v})$ to
$(\vec{q},\vec{p})$ using position and momenta.
At the end of this step, we should have obtained the velocities as a
function of positions and momenta
\begin{equation}
v^{j} := v^{j}(\vec{q},\vec{p}).
\end{equation}

\begin{remark}
Usually this is
possible. We check by taking the determinant of the Jacobian matrix
\begin{equation}
\det(\partial p_{i}(\vec{q},\vec{v})/\partial v^{j})\neq0.
\end{equation}
But there are times when the determinant of the Jacobian matrix equals
zero. This means there is some redundancy in the definition of momenta.
The solution is to introduce extra conditions called \define{Primary Constraints}
$\phi_{m}(\vec{q},\vec{p})=0$. This is rather complicated, the idea
being these constraints are used as coordinates to describe the rest of
the phase space. The curious reader may consult Henneaux and
Teitelboim's \emph{Quantization of Gauge Systems} (the first chapter
alone suffices).
\end{remark}

\N{Step three: construct the Hamiltonian}
We may then construct the Hamiltonian using the \define{Legendre transform}:
\begin{equation}\label{eq:hamiltonian-def}
H(\vec{q},\vec{p}) = \left(\sum^{n}_{j=1}p_{j}v^{j}(\vec{q},\vec{p})\right) - L(\vec{q},\vec{v}(\vec{q},\vec{p})).
\end{equation}
This plays the same role in Hamiltonian mechanics as the Lagrangian
plays in Lagrangian mechanics. There may be explicit time dependence, in
which case
\begin{equation}
H(\vec{q},\vec{p},t) = \left(\sum^{n}_{j=1}p_{j}v^{j}(\vec{q},\vec{p})\right) - L(\vec{q},\vec{v}(\vec{q},\vec{p}),t).
\end{equation}

\begin{remark}[On the Legendre Transform]
The Legendre transform sounds exotic, but it's just a way to transform
from a convex function to a concave function (and
vice-versa). Specifically, for physics (where we have a particular
function of interest, like the Lagrangian), when we have a set of
coordinates $(q,v)$ and we want to change to a different set of
coordinates $(q,p)$ where $v$ and $p$ are conjugate variables, then we
use the Legendre transform to translate our function of interest. It is
always of the form
\begin{equation}
\begin{pmatrix}\mbox{new}\\
\mbox{function}
\end{pmatrix}
=\begin{pmatrix}
\mbox{variable}
\end{pmatrix}\begin{pmatrix}
\mbox{conjugate}\\
\mbox{variable}
\end{pmatrix}
-
\begin{pmatrix}
\mbox{original}\\
\mbox{function}
\end{pmatrix}.
\end{equation}
If the reader continues to study thermodynamics, the Legendre transform
returns in that setting as well.
\end{remark}

\begin{exercise}
Prove, if $L=K-U$ the Lagrangian is just the difference between the
kinetic energy $K$ and potential energy $U$, then $H=K+U$ the
Hamiltonian is the total energy.
\end{exercise}

\begin{exercise}
Prove this process is invertible, namely: if we begin with $H = K(q,p) + U(q,p)$,
then change coordinates to the velocity $p=p(q,v)$, then
$L = 2K(q,p(q,v)) - H(q,p(q,v))$ is the original Lagrangian.
\end{exercise}

\N{Deriving the equations of motion}
There are two ways to derive the equations of motion for Hamiltonian
mechanics. The first is to algebraically manipulate the definition of
the Hamiltonian in Eq~\eqref{eq:hamiltonian-def} to write
\begin{equation}
L(\vec{q},\vec{v}(\vec{q},\vec{p})) = \left(\sum^{n}_{j=1}p_{j}v^{j}(\vec{q},\vec{p})\right) - H(\vec{q},\vec{p}).
\end{equation}
We plug this into the action integral, apply the principle of stationary
action, then compare the coefficients of $\delta q$ and $\delta p$. Then
we can infer what the Euler--Lagrange equations should be in terms of
Hamiltonians.

This is messy, somewhat unmotivated, and error prone. A slicker
approach: we write the Hamiltonian as a function of positions, velocities, and
momenta variables (all independent of each other):
\begin{equation}
H(\vec{q},\vec{v},\vec{p}) = \vec{v}\cdot\vec{p} - L(\vec{q},\vec{v}).
\end{equation}
Then we have
\begin{equation}
v^{j} = \frac{\partial H}{\partial p^{j}}
\end{equation}
which we replace the left-hand side with the time-derivative of
position, and any instances of velocity in the right-hand side with the functions
$v^{j}(\vec{q},\vec{p})$ to get the equations of motion for positions,
\begin{subequations}\label{eq:cheat:eom}
\begin{equation}
\frac{\D}{\D t}q^{j} = \frac{\partial H(\vec{q},\vec{p})}{\partial p_{j}},
\end{equation}
and from Euler--Lagrange we obtain the equations of motion for momenta,
\begin{equation}
\frac{\D}{\D t}p_{j} = -\frac{\partial H(\vec{q},\vec{p})}{\partial q^{j}}.
\end{equation}
\end{subequations}
Now, as with Lagrangian mechanics, we make the positions and momenta
some unknown functions of time, then Eq~\eqref{eq:cheat:eom} give us our
equations of motion as first-order differential equations.


\begin{exercise}
  For one-dimensional simple Harmonic oscillator, we have
  \begin{equation}
H = \frac{1}{2m}p^{2} + \omega^{2}q^{2}.
  \end{equation}
Determine the equations of motion and solve them.
\end{exercise}

\begin{remark}
One of the hallmarks of analytical mechanics is this seemingly arbitrary
decision to treat quantities as independent variables, and other times
as functions of the other variables. This sadly cannot be avoided, and
only becomes understandable after working with them for a long time.
\end{remark}

\N{Poisson Bracket}
The other approach is less motivated but more elegant. We introduce a
mathematical gadget called the \define{Poisson Bracket}. We define it as
operating on two functions on the phase space $f$ and $g$ by the equation:
\begin{equation}
  \{f(\vec{q},\vec{p}), g(\vec{q},\vec{p})\}
  := \sum_{j}\frac{\partial f(\vec{q},\vec{p})}{\partial q^{j}}\frac{\partial g(\vec{q},\vec{p})}{\partial p_{j}}
  -\frac{\partial g(\vec{q},\vec{p})}{\partial q^{j}}\frac{\partial f(\vec{q},\vec{p})}{\partial p_{j}}.
\end{equation}
This seems useless, right? Well, look at Eq~\eqref{eq:cheat:eom} again.
What happens if we plug in $f=H$ to the Poisson bracket? We get:
\begin{equation}
  \{H(\vec{q},\vec{p}), g(\vec{q},\vec{p})\}
  = \sum_{j}\frac{\partial H(\vec{q},\vec{p})}{\partial q^{j}}\frac{\partial g(\vec{q},\vec{p})}{\partial p_{j}}
  -\frac{\partial g(\vec{q},\vec{p})}{\partial q^{j}}\frac{\partial H(\vec{q},\vec{p})}{\partial p_{j}}.
\end{equation}
If we then make $g$ just a component of position $q^{i}$, then we get
\begin{subequations}
\begin{align}
  \{H(\vec{q},\vec{p}), q^{i}\}
  &= \sum_{j}\frac{\partial H(\vec{q},\vec{p})}{\partial q^{j}}\frac{\partial q^{i}}{\partial p_{j}}
  -\frac{\partial q^{i}}{\partial q^{j}}\frac{\partial H(\vec{q},\vec{p})}{\partial p_{j}}\\
  &= \sum_{j}\frac{\partial H(\vec{q},\vec{p})}{\partial q^{j}}0
  -{\delta^{i}}_{j}\frac{\partial H(\vec{q},\vec{p})}{\partial p_{j}}\\
  &= -\frac{\partial H(\vec{q},\vec{p})}{\partial p_{i}}.
\end{align}
\end{subequations}
This is precisely the equation of motion for position:
\begin{equation}
\frac{\D}{\D t}q^{i} = -\{H(\vec{q},\vec{p}), q^{i}\}.
\end{equation}
We may perform similar computation to find the Poisson bracket of the
Hamiltonian with momenta gives us the equations of motion for momentum
\begin{equation}
\frac{\D}{\D t}p_{i} = -\{H(\vec{q},\vec{p}), p_{i}\}.
\end{equation}
The astute reader will realize
\begin{equation}
\{f(\vec{q},\vec{p}), g(\vec{q},\vec{p})\}
=-\{g(\vec{q},\vec{p}), f(\vec{q},\vec{p})\}.
\end{equation}
We can use this to rewrite the equations of motion with the Hamiltonian
in the second slot, and no minus signs needed.

\begin{definition}
We call coordinates $q_{i}$, $p_{j}$ on the phase
space \define{Canonical Coordinates} (or \emph{Darboux coordinates}) if,
for any $i$, $j$, we have:
\begin{subequations}
\begin{align}
\{q^{i}, q^{j}\} &= 0\\
\{p_{i}, p_{j}\} &= 0\\
\{q^{i}, p_{j}\} &= {\delta^{i}}_{j}
\end{align}
\end{subequations}
where ${\delta^{i}}_{j}=0$ if $i\neq j$ and ${\delta^{i}}_{j}=1$ when $i=j$
(i.e., it's the Kronecker delta).
\end{definition}

\begin{remark}
There is a theorem, due to Darboux, establishing for any
finite-dimensional phase space, there always exists locally (i.e.,
within a patch diffeomorphic to an open ball) canonical coordinates.
This theorem fails to hold for field theories. Further, the phase space
may be covered by such patches with Darboux coordinates.
\end{remark}

\M
More generally, we have the equations of motion for any function $F$ of
the phase space be given by its Poisson bracket with the Hamiltonian:
\begin{equation}\label{eq:eom-from-poisson-bracket}
\frac{\D}{\D t}F = \{F, H\}.
\end{equation}
This gives us the equations of motion \emph{for any quantity}. If $F$
explicitly depends on time --- i.e., looks like $F(\vec{q},\vec{p},t)$
--- then the equations of motion look like:
\begin{equation}
\frac{\D}{\D t}F = \{F, H\} + \frac{\partial F}{\partial t}.
\end{equation}
When $F$ does not have explicit time dependence, the partial derivative
with respect to time vanishes, recovering Eq~\eqref{eq:eom-from-poisson-bracket}.

We also avoid this tedium of ``pretend velocities are variables, do this
calculation, now pretend the velocities are functions of momenta''.
Instead, we can jump from writing down the Hamiltonian straight to the
equations of motion.

\begin{exercise}
Prove the Poisson bracket is antisymmetric; i.e., for any phase-space
functions $f$ and $g$,
\begin{equation*}
\{f,g\} = -\{g,f\}.
\end{equation*}
\end{exercise}

\begin{exercise}[Jacobi Identity]
For any phase space functions $f$, $g$, $h$, prove:
\begin{equation*}
\{f,\{g,h\}\} + \{g,\{h,f\}\} + \{h,\{f,g\}\} = 0.
\end{equation*}
This is called the \define{Jacobi Identity}. Together with the previous
exercise, we have proven the set of functions on the same phase space
--- when equipped with the Poisson bracket --- form a Lie algebra.
\end{exercise}

\begin{exercise}
  For one-dimensional simple Harmonic oscillator, we have
  \begin{equation}
H = \frac{1}{2m}p^{2} + \omega^{2}q^{2},
  \end{equation}
  where $g>0$ is constant.
Compute the Poisson bracket $\{-,H\}$ leaving the first slot free.
\end{exercise}

\begin{exercise}
  For one-dimensional gravitational well, let $g>0$ be constant, we have
  \begin{equation}
H = \frac{1}{2m}p^{2} + \frac{g}{q}.
  \end{equation}
Compute the Poisson bracket $\{-,H\}$ leaving the first slot free.
\end{exercise}

\subsection{Sketches of Variational Arguments}

\N{NOTE/CAUTION} We will be using Einstein summation convention here,
so if an index appears once upstairs and once downstairs, we sum over it.
We will also treat the variation $\delta$ operator as a differential
operator (the technical word is \emph{linear derivation}). Physicists
will be comfortable here, mathematicians may experience some discomfort.



\begin{theorem}\label{thm:Hamiltonian-depends-on-p-and-q}
The variation of the Hamiltonian is a linear combination of variations
in momentum and position; in particular,
\begin{equation}\label{eq:variation-of-H}
\delta H = v^{j}\cdot\delta p_{j} - \frac{\partial L}{\partial q^{j}}\delta q^{j}.
\end{equation}
\end{theorem}

\begin{proof}
  We find this directly by:
  \begin{subequations}
    \begin{align}
\delta H
&= \delta(v^{j}p_{j} - L)\\
&= p_{j}\cdot\delta v^{j} + v^{j}\cdot\delta p_{j} -
\left(\frac{\partial L}{\partial q^{j}}\delta q^{j} + \frac{\partial L}{\partial v^{j}}\delta v^{j}\right)\\
&= \left(p_{j} - \frac{\partial L}{\partial v^{j}}\right)\delta v^{j}
+ v^{j}\cdot\delta p_{j} - \frac{\partial L}{\partial q^{j}}\delta q^{j}\\
&= (0)\delta v^{j} + v^{j}\cdot\delta p_{j} - \frac{\partial L}{\partial q^{j}}\delta q^{j}\\
&= v^{j}\cdot\delta p_{j} - \frac{\partial L}{\partial q^{j}}\delta q^{j}.
    \end{align}
  \end{subequations}
Hence there is no way for $\delta v^{j}$ to enter, implying $H=H(q,p)$
is a function of positions and momenta.
\end{proof}


\begin{lemma}\label{lemma:variation-of-kinetic-contribution}
  The variation of the kinetic term $v^{j}p_{j}$ may be written as:
  \begin{equation}
\delta(v^{j}p_{j}) = v^{j}\cdot\delta p_{j} -\frac{\partial L}{\partial q^{j}}\delta q^{j}.
  \end{equation}
\end{lemma}

\begin{proof}
  We can see this by first using the product rule
    \begin{equation}
\delta(v^{j}p_{j}) = v^{j}\cdot\delta p_{j} + p_{j}\cdot\delta v^{j}.
  \end{equation}
Then we use the fact this will appear in an integral, so we can do the
following dodgy physicist's trick: replace $\delta v^{j}$ with the time
derivative of $\delta q^{j}$, then
integrate by parts and (per tradition of physicists) ignore the boundary
term to get
\begin{equation}
p_{j}\cdot\delta v^{j} = -\frac{\D p_{j}}{\D t}\delta q^{j}.
\end{equation}
We rewrite the right-hand side using the Euler-Lagrange equations as
\begin{equation}
\frac{\D p_{j}}{\D t}\delta q^{j} = \frac{\partial L}{\partial q^{j}}\delta q^{j}.
\end{equation}
Then putting things back together, we get
\begin{equation}
\delta(v^{j}p_{j}) = v^{j}\cdot\delta p_{j} - \frac{\partial L}{\partial q^{j}}\delta q^{j}.
\end{equation}
This proves the result.
\end{proof}


\begin{theorem}
  The variation of the Lagrangian, in canonical coordinates, is given by
  \begin{equation}
\delta L = \left(\frac{\partial H}{\partial p_{j}}-v^{j}\right)\delta p_{j} -
\left(\frac{\partial H}{\partial q^{j}} + \frac{\partial L}{\partial q^{j}}\right)\delta q^{j}.
  \end{equation}
\end{theorem}

\begin{proof}
  We begin by observing
  \begin{equation}
L = v^{j}p_{j} - H.
  \end{equation}
  Then we find
  \begin{equation}
\delta L = \delta(v^{j}p_{j} - H).
  \end{equation}
  We use Theorem~\ref{thm:Hamiltonian-depends-on-p-and-q} and take the variation of the Hamiltonian as a linear combination of
  variations in position and momentum:
  \begin{equation}
\delta H = \frac{\partial H}{\partial p_{j}}\delta p_{j} +
\frac{\partial H}{\partial q^{j}}\delta q^{j}.
  \end{equation}
  Then using Lemma~\ref{lemma:variation-of-kinetic-contribution}, we find
  \begin{equation}
\delta(v^{j}p_{j}) =  v^{j}\cdot\delta p_{j} -\frac{\partial L}{\partial q^{j}}\delta q^{j}.
  \end{equation}
  Combining these two results, we find:
  \begin{subequations}
 \begin{align}
\delta L &= v^{j}\cdot\delta p_{j} -\frac{\partial L}{\partial
  q^{j}}\delta q^{j} - \left(\frac{\partial H}{\partial p_{j}}\delta p_{j} +
\frac{\partial H}{\partial q^{j}}\delta q^{j}\right)\\
&= \left(v^{j} - \frac{\partial H}{\partial p_{j}}\right)\delta p_{j} +
\left(-\frac{\partial L}{\partial q^{j}}-\frac{\partial H}{\partial q^{j}}\right)\delta q^{j}.
 \end{align}
  \end{subequations}
  This proves the claim.
\end{proof}


\begin{corollary}[Equations of motion]
The fundamental lemma of variational calculus implies the variation of
the action vanishes only when $\delta L$ vanishes, which requires the
coefficients of $\delta q^{j}$ and $\delta p_{j}$ independently must
vanish. This gives us the equations
\begin{equation}
v^{j} = \frac{\D}{\D t}q^{j} = \frac{\partial H}{\partial p_{j}}
\end{equation}
(i.e., the equation of motion for $q^{j}$) and
\begin{equation}
\frac{\D}{\D t}p_{j} = \frac{\partial L}{\partial q^{j}}=-\frac{\partial H}{\partial q^{j}}
\end{equation}
(i.e., the equation of motion for $p_{j}$).
\end{corollary}

\M
We may confer with Eq~\eqref{eq:cheat:eom} to find, in fact, we have
obtained the equations of motion. This Kabuki theater may be
unenlightening, but it demonstrates the care which must be considered
when performing variational calculus in the Hamiltonian formalism. Also
notice, we did needless computation. We could have jumped straight from
Theorem~\ref{thm:Hamiltonian-depends-on-p-and-q} to the equations of
motion by taking
\begin{equation}
\delta H = \frac{\partial H}{\partial p_{j}}\delta p_{j} +
\frac{\partial H}{\partial q^{j}}\delta q^{j},
\end{equation}
then comparing the coefficients of $\delta q^{j}$ and $\delta p_{j}$.
On the other hand, this can be used as a \emph{consistency result},
a way to evaluate our theorems and corollary, to make sure they make
sense.

\subsection{So What?}

\M
There is this lingering air of \emph{a result has been proven but its significance seems elusive}.
In short, we care because we can obtain a description of a physical
system which is consistent with our other ways to describe physical
systems. We obtain the equations of motion in a way coherent with our
pre-existing techniques. We also obtain a ``shortcut'' to the equations
of motion.

But why should we care? Because almost every physical system we care
about can be cast in the Hamiltonian mould. And regardless of choice of
coordinates, they always have the same formulas for the equations of
motion. For these reasons, the only freedom we have is in choice of
Hamiltonian as a scalar function of the phase space. Once it is
determined, everything we do is (in some sense) ``canonical'' or
immediately obvious. This is presumably why Carl Jacobi referred to Hamiltonian
mechanics as \emph{Canonical Mechanics}.

\N{Geometry}
There are \emph{geometric} reasons for coordinate independence, but this
discussion would require knowing quite a bit of differential geometry.
We can encode the Poisson bracket with one slot ``free'', the second
slot being the Hamiltonian, as a vector field on the configuration
space. This can be transformed using musical isomorphisms into a
differential form. The choice of differential form is determined by the
topology of the phase space (thanks to cohomology), which means we have
only certain specific choices.

There is much beauty to be admired here, and it is further studied in
the field of \emph{symplectic geometry}.

\N{Quantum Theory}
It is impossible to do quantum mechanics without Hamiltonian mechanics.
The basic idea is to consider some ``translator'' mapping
\begin{equation}
  \mathcal{Q}\colon\begin{pmatrix}\mbox{Classical}\\
  \mbox{Mechanics}
  \end{pmatrix}\to\begin{pmatrix}\mbox{Quantum}\\
  \mbox{Mechanics}
  \end{pmatrix}
\end{equation}
such that the Poisson bracket becomes a commutator:
\begin{equation}
\mathcal{Q}(\{A,B\}) = [\mathcal{Q}(A), \mathcal{Q}(B)]/(\I\hbar).
\end{equation}
Usually use a ``hat'' instead of $\mathcal{Q}(-)$, and a particularly
convenient choice is that
\begin{equation}
\mathcal{Q}(q^{j}) = q^{j}
\end{equation}
and
\begin{equation}
\mathcal{Q}(p_{j}) = \I\hbar\frac{\partial}{\partial q^{j}}.
\end{equation}
This works with the demand the Poisson bracket becomes a commutator.
There are other possible choices, this coincides with the Schrodinger picture.

It turns out, well, it's impossible for such a mapping to exist
unambiguously (this is the van Hove--Groenewald ``no go'' theorem). But
you can't appreciate this without knowing Hamiltonian mechanics!
See Mark Gotay's ``Obstructions to Quantization'', \arXiv[math-ph]{9809011},
for further details.


\begin{exercise}
Prove, in one dimension, we have:
\begin{equation}
q^{2}p^{2} = \frac{1}{9}\{q^{3},p^{3}\} = \frac{1}{3}\{q^{2}p,qp^{2}\}.
\end{equation}
\end{exercise}

\begin{exercise}
Show
\begin{equation}
\mathcal{Q}\left(\frac{1}{9}\{q^{3},p^{3}\}\right)
\neq \mathcal{Q}\left(\frac{1}{3}\{q^{2}p,qp^{2}\}\right).
\end{equation}
In particular, there is ambiguity when computing $\mathcal{Q}(q^{2}p^{2})$.
\end{exercise}


\subsection{Gauge Theory Briefly}

\M
The basic idea is when we have a singular Jacobian matrix
\begin{equation}
\det(\partial p_{k}/\partial v^{j})=0,
\end{equation}
we have some redundancy among the momenta. If
\begin{equation}
\mathrm{rank}(\partial p_{k}/\partial v^{j})=M,
\end{equation}
then (if we have $N$ momenta coordinates in the phase space) we need $N-M$ additional conditions called \define{Primary Constraints}.
Geometrically, what's going on is each value of a velocity $v^{j}$
corresponding to a ``redundant'' momentum --- this is transformed to the
entire dimension of its conjugate momenta $p_{j}$, since its
underspecified. We fix this freedom using Lagrange multipliers.
These define a surface in the phase space called the \define{Constraint Surface}.

We denote the primary constraints by $\phi_{k}(\vec{q},\vec{p})$.

\M
Basically, the idea is we do physics on the constraint surface, and fix
coordinates on the constraint surface using Lagrange multipliers. The
constraints need to satisfy some consistency conditions (intuitively
they impose the condition they are constants of motion). This is
precisely what happens mathematically with gauge theory: not all the
constraints are necessary (some are redundant), which injects some
arbitrary linear combination of them. This arbitrariness needs to be
fixed using gauge conditions, and so on.

\section{Coordinates}

\M
We take advantage of coordinate independence to choose some exotic
coordinates. We refer to changes of coordinates
\begin{equation}
(\vec{q}, \vec{p},t)\to(\vec{Q},\vec{P},t)
\end{equation}
as \define{Canonical transformations} provided they preserve the
equations of motion. Undergraduates are typically tasked with questions
along the lines of ``Given $\vec{q}\to\vec{Q}$, determine the new
momenta and Hamiltonian.''

\N{Action--Angle Coordinates}
For example, we can write the action as:
\begin{equation}
S = \int p_{j}\,\D q^{j} - \int H\,\D t.
\end{equation}
We can pick new coordinates which resemble actions:
\begin{equation}
J_{j} := \int p_{j}\,\D q^{j}.
\end{equation}
Implicitly, the path of integration takes place along contours of
constant energy $E=E(q^{k}, p_{k})$.
We interpret $J_{j}$ as ``momenta''. Its associated positions are called
\define{Angles} denoted $\theta_{j}$ (hence the name ``Action--Angle coordinates'').

If we can transform the Hamiltonian to be
\begin{equation}
H(\vec{q},\vec{p}) = K(\vec{J}),
\end{equation}
then we find
\begin{equation}
\dot{J}_{k} = -\frac{\partial K}{\partial\theta_{k}} = 0.
\end{equation}
Hence the action-coordinates are constants of motion. This is possible
if and only if the system is integrable.

\begin{exercise}
  For one-dimensional simple Harmonic oscillator, we have
  \begin{equation}
H = \frac{1}{2m}p^{2} + \omega^{2}q^{2}.
  \end{equation}
(The superscripts are powers, not indices.) Determine the action--angle
  coordinates for this system.
\end{exercise}

\begin{remark}[Importance: ``Old'' Quantum Theory]
Historically, quantum mechanics began by quantizing the action
coordinates, so
\begin{equation}
J_{k} = 2\pi n_{k}\hbar
\end{equation}
for some $n_{k}\in\ZZ$. This is known as the Bohr--Sommerfeld condition.
\end{remark}

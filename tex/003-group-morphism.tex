\M
There is some ``Kabuki theater'' whenever introducing a new mathematical
gadget, thanks to \ifblog\href{https://texnicalstuff.blogspot.com/2009/08/object-oriented-math-category-theory-as.html}{category theory}:\fi\iftex category theory:\fi
we have our new gadget, then we could ask about morphisms (very
important), ``subgadgets'', and universal constructions (products,
quotients, etc.). The exciting thing, as in Kabuki theater, is the order
and manner of presentation.

\begin{definition}
  Let $G$ and $H$ be groups. We define a \define{Group Morphism} to
  consist of a function $\varphi\colon G\to H$ of the underlying sets
  such that
  \begin{enumerate}
  \item group operation is preserved: for any $g_{1}$, $g_{2}\in G$, we have $\varphi(g_{1}g_{2})=\varphi(g_{1})\varphi(g_{2})$;
  \item identity element is preserved: if $e_{G}\in G$ is the identity
    element of $G$ and $e_{H}\in H$ is the identity element of $H$, then $\varphi(e_{G})=e_{H}$;
  \item inverse is preserved: for any $g\in G$, $\varphi(g^{-1})=\varphi(g)^{-1}$.
  \end{enumerate}
\end{definition}

\begin{remark}
  Older texts refer to group morphisms as ``group
  \emph{homomorphisms}''.
  After category theory became popular and part of the standard
  curriculum, the ``homo-'' prefix was dropped because group morphisms
  live in the same category, so it was redundant.
\end{remark}

\begin{example}
  One ``low hanging fruit'' for morphism examples is the identity
  morphism. We should check, for any group $G$, the identity function
  $\id\colon G\to G$ is a \emph{bona fide} group morphism.

  We see it preserves the group operation. For any $g_{1}$, $g_{2}\in G$,
  we have $\id(g_{1}g_{2})=g_{1}g_{2}$ by definition of the identity
  function. But this is also equal to $\id(g_{1})\id(g_{2})$. Thus the
  group operation is preserved.

  The identity element is preserved $\id(e_{G})=e_{G}$.

  Inversion is also preserved $\id(g^{-1})=g^{-1}=\id(g)^{-1}$ for any
  $g\in G$.

  Thus taken together, it follows the identity mapping satisfies the
  axioms of a group morphism.
\end{example}

\begin{example}
  Let $G$ be any group, and consider $\ZZ$ equipped with addition as a
  group. For each $g\in G$, we have a group morphism
  $\varphi\colon\ZZ\to G$ sending $1\in\ZZ$ to $g\in G$. Is this
  \emph{really} a group morphism?

  We can check that the properties are (or, ought to be) satisfied. If
  the group operation is preserved, then
  $\varphi(1+1)=\varphi(1)\varphi(1)=g^{2}$ and more generally, for any
  $m\in\ZZ$, we have
  $\varphi(m+1)=\varphi(1)^{m}=g^{m}$.

  For the identity element being preserved, that means
  $\varphi(0)=e_{G}$, which is fine: it corresponds to $g^{0}=e_{G}$.

  Group inverses would be
  $\varphi(-m)=\varphi(m)^{-1}=(g^{m})^{-1}$. And we know this is
  precisely the same as $g^{-m}$.
\end{example}

\begin{example}
  Consider the group $\GL(2,\RR)$ and the multiplicative group
  $\RR^{\times}$ of nonzero real numbers. Then the determinant
  \begin{equation}
    \det\colon\GL(2,\RR)\to\RR^{\times}
  \end{equation}
  is a group morphism. Let us prove it!

  We see, for any matrices $M$, $N\in\GL(2,\RR)$ we have
  \begin{equation}
    \det(MN)=\det(M)\det(N).
  \end{equation}
  This is a familiar fact in linear algebra. But for us, it tells us the
  group operation is preserved.

  The identity element must be mapped to the identity element. We see
  the identity matrix $I\in\GL(2,\RR)$ has $\det(I)=1$. Thus the
  determinant preserves the group identity element.

  As far as the group inverse, well, this follows from previous results,
  right? After all, if $M\in\GL(2,\RR)$, then $M^{-1}\in\GL(2,\RR)$, and
  \begin{equation}
    I = M^{-1}M
  \end{equation}
  so
  \begin{equation}
    \det(M^{-1}M)=\det(M^{-1})\det(M)=1
  \end{equation}
  and thus by division
  \begin{equation}
    \det(M^{-1})=\det(M)^{-1}.
  \end{equation}
  Thus the group inverse operator is preserved.
\end{example}

\begin{definition}
  Let $\varphi\colon G\to H$ be a group morphism. We define the
  \define{Kernel} of $\varphi$ to be the pre-image of the identity
  element of $H$:
  \begin{equation*}
    \ker(\varphi)=\{g\in G|\varphi(g)=e_{H}\}.
  \end{equation*}
\end{definition}

\begin{example}
  For the group morphism $\det\colon\GL(2,\RR)\to\RR^{\times}$, the
  kernel would be
  \begin{equation}
    \ker(\det)=\{M\in\GL(2,\RR)|\det(M)=1\}.
  \end{equation}
  That is to say, it consists of matrices with unit
  determinant. Observe, this is a group under matrix multiplication: if
  two matrices have unit determinant, their product has unit
  determinant; the identity matrix is in the kernel; and it's closed
  under inverses. This is an important group called the
  \define{Special Linear Group}, denoted $\SL(2,\RR)$.
\end{example}

\section{Properties of Morphisms}

\begin{proposition}
  Let $\varphi\colon G\to H$ be a group morphism.
  If $g\in G$ is any element, then $\varphi(g^{-1})=\varphi(g)^{-1}$.
\end{proposition}

\begin{proof}
  Let $g\in G$ (so $\varphi(g)\in H$).
  We find $\varphi(g\cdot g^{-1})=\varphi(g)\varphi(g^{-1})=e_{H}$,
  thus multiplying on the left by $\varphi(g)^{-1}$ gives the result.
\end{proof}

\begin{proposition}
  Let $\varphi\colon G\to H$ be a group morphism.
  If $g\in G$ is any element and $n\in\ZZ$ is any integer,
  then $\varphi(g^{n})=\varphi(g)^{n}$.
\end{proposition}

\begin{proof}
  Per cases since $n<0$ or $n=0$ or $n>0$. The $n=0$ case is obvious.

  For $n>0$, by induction. The base case $n=1$ gives
  $\varphi(g^{1})=\varphi(g)^{1}$, which is obvious. Assume this holds
  for arbitrary $n$. Then the inductive case $n+1$ is
  \begin{equation}
    \varphi(g^{n+1})=\varphi(g^{n}g)=\varphi(g)^{n}\varphi(g)=\varphi(g)^{n+1}.
  \end{equation}
  Thus we have proven the result for non-negative $n$.

  For negative $n\in\ZZ$, the proof is analogous.
\end{proof}

\begin{theorem}
  The composition of group morphisms is a group morphism. More
  explicitly, if $\varphi\colon G\to H$ and $\psi\colon H\to K$ are
  group morphisms, then $\psi\circ\varphi\colon G\to K$ is a group morphism.
\end{theorem}

\begin{theorem}
  Let $\varphi\colon G\to H$ be a group morphism.
  If $\ker(\varphi)=\{e_{G}\}$, then $\varphi$ is injective.
\end{theorem}
\begin{proof}
  Assume $\ker(\varphi)=\{e_{G}\}$.
  Let $g_{1}$, $g_{2}\in G$ be completely arbitrary. (We want to show if
  $\varphi(g_{1})=\varphi(g_{2})$, then $g_{1}=g_{2}$.)
  Assume $\varphi(g_{1})=\varphi(g_{2})$.
  Then $\varphi(g_{1})\varphi(g_{2})^{-1}=e_{H}$ by multiplying both
  sides on the right by $\varphi(g_{2})^{-1}$.
  And $\varphi(g_{1})\varphi(g_{2}^{-1})=\varphi(g_{1}\cdot g_{2}^{-1})=e_{H}$.
  Thus $g_{1}\cdot g_{2}^{-1}\in\ker(\varphi)$ by definition of the
  kernel.
  But we assumed the only member of the kernel was identity element.
  Thus $g_{1}\cdot g_{2}^{-1}=e_{G}$, and moreover $g_{1}=g_{2}$.
  Hence $\varphi$ is injective.
\end{proof}

\section{Exercises}

\begin{exercise}
  Let $G$ be a group, $n\in\NN$ be a fixed positive integer.
  Prove or find a counter-example: $\varphi\colon G\to G$, sending $g$
  to $\varphi(g)=g^{n}$ is a group morphism.
\end{exercise}

\begin{exercise}
  Prove or find a counter-example: the matrix trace
  $\tr\colon\GL(2,\RR)\to\RR$ is a group morphism.
\end{exercise}

\begin{exercise}
  Is the exponential function on the real numbers a group morphism $\exp\colon\RR\to\RR^{\times}$?
\end{exercise}

\begin{exercise}
  Is the matrix exponential a group morphism $\exp\colon\Mat_{2}(\RR)\to\GL(2,\RR)$?
\end{exercise}

